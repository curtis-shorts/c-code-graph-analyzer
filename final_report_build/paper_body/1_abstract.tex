\begin{abstract}
%\cus{200 word limit}
High Performance Computing (HPC) communication libraries utilize network middleware libraries in order to minimize the latency experienced when interacting with the networking hardware that makes large-scale cluster computing possible. The issue for new developers of these middleware libraries is that they have little-to-no developer-level documentation on their architectures and are written in C, the semantics of which can make them difficult to interpret at a high level. This paper proposes a methodology for analyzing the architecture of HPC middleware libraries by parsing the source-code to form a weighted dependency graph, then extracting the libraries' architecture with clustering algorithms. The parsing techniques used in this work use a combination of regular expressions (RegEx) and ANTLR, a parser generation tool. RegEx and ANTLR allowed for simple detection of pre-processor declarations (i.e. macros) and context-dependent usage of more complex dependencies (e.g. functions and structs) respectively. The results presented in this work showed the dependency graphs, but not the clustering of the graphs, to be useful in analyzing the source-code of such libraries through developing an initial intuition as to key architectural features.
\end{abstract}
