\section{Conclusions and Future Work}
\label{sec:conclusions}

This work presented a methodology for extracting dependency graphs from code-bases written in C for the purposes of source code architecture extraction to further new developers understanding of complex codes. The methodology included a combination of ANTLR and RegEx to extract dependency graph, two algorithms for cluster detection in the graph using weighted and unweighted heuristics, and the presentation of the data in network diagrams that display the code-bases topology in a hierarchical manner for ease of analysis. The applications explored through this workflow were specific to the area of HPC middlewares as they are notorious for high levels of complexity and a lack of documentation, but the generalization of the solution was tested in preliminary to ensure the methodology was sound for all C-based codes. The results of testing the methodology proved that the clustering algorithms did not reveal any useful information about the underlying code-bases, but the network diagram representations of the weighted graphs were very helpful in the analysis of the code-bases as was seen through the in-depth analysis of one of the test codes.

Future work should seek to look at alternative methodologies for architecture extraction as the hierarchical clustering component presented in this work did not produce insightful results. Possible alternative methodologies in order of decreasing similarity to this work include trying different hierarchical organization methods to represent the codes differently for algorithm input, using a neural network instead of traditional algorithms to see if alternative patterns can be detected, and using an LLM-based workflow for more context-aware architecture analysis. Since the network diagrams were useful for the analysis of codes, further development should also be sought in that avenue. A tool with an interactive GUI could be developed to allow a better exploration of the weighted dependencies. The specific features such a tool could provide that isn't available in existing solutions would need to be evaluated in more depth, but the general value proposition could be a new way to view code hierarchies to give more context as their weighted dependency structure.

%but the generally offering could be to provide more context to the codes under study and allow the user to put the hierarchies in more context of the overall code-base than can be represented in a static image.

%This tool could provide more context to the codes under study and allow the user to put the hierarchies in more context of the overall code-base than can be represented in a static image.

%This paper outlined a course project proposal for ELEC 876. The proposed project aims to gain an understanding of how UCX, a middleware framework for network communication standards in HPC, is structured and operates. This understanding will be gained by generating an AST of the component pieces of UCX, and UCX in it's entirety, using C-language lexers and parsers generated with ANTLR and a custom parsing script. The AST will then be analyzed using clustering algorithms to traverse the tree and identify patterns within the code, both at the component and holistic levels. The intuition gained from the pattern identification along with the other representations of UCX available will be used to develop some form of documentation for how the code is structured and operates. This will be useful to many people in the HPC community who develop application interfaces such as MPI as it will give them a better understanding about how the middleware is used to interact with the network hardware.
